\documentclass[20pt,twoside]{memoir}

\ifx\EnableEbookVersion\undefined
    \usepackage[osf]{libertine} % Use the Libertine font
\fi



\usepackage{lipsum}
\usepackage[utf8]{inputenc}
\usepackage{graphicx} % Pacote para incluir gráficos
\usepackage{tikz} % Pacote para gráficos vetoriais
\usepackage{xcolor} % Pacote para cores

\usepackage{geometry}
\geometry{paperwidth=209.6mm,paperheight=209.6mm, % Tamanho A4
          left=20mm,top=20mm,                 % Margens
          right=20mm,bottom=20mm}


% Definir a cor global da fonte
\definecolor{mycolor}{RGB}{0,90,150} % Defina a cor desejada (ex: azul)

% Comando para definir a imagem de fundo
\newcommand{\backgroundimage}[1]{
    \begin{tikzpicture}[remember picture,overlay]
        \node[at=(current page.south west),anchor=south west,inner sep=0pt] {
            \includegraphics[width=\paperwidth,height=\paperheight,keepaspectratio]{#1}
        };
    \end{tikzpicture}
}

%\renewcommand{\normalsize}{\fontsize{28pt}{34pt}\selectfont}

\newenvironment{slide-frame}
{
\clearpage % Começar em uma nova página
%\noindent
}
{
\clearpage % Terminar com uma nova página
}

\newenvironment{slide-frame-center}%
{
\clearpage % Começar em uma nova página
\vspace*{\fill}%
\centering % Alinhamento horizontal
%%
}
{
\vspace*{\fill}%
\clearpage % Terminar com uma nova página
}


\input{structure/book-box/conf1}

\begin{document}

\begin{slide-frame-center}
\noindent
Não fui eu que ordenei a você? Seja forte e corajoso! Não se apavore nem desanime, 
pois o Senhor, o seu Deus, estará com você por onde você andar
\end{slide-frame-center}

\begin{slide-frame}
\centering
\includegraphics[width=0.7\textwidth]{path/to/your/image.eps}
\end{slide-frame}

\begin{slide-frame}
\begin{book-box-top}texto com cor global\end{book-box-top}
\lipsum[1][1-2]
\end{slide-frame}

\begin{slide-frame}
\backgroundimage{path/to/your/background1.jpg} % Substitua pelo caminho da sua imagem
\begin{book-box-top}texto com cor global\end{book-box-top}
\centering
\includegraphics[width=0.7\textwidth]{path/to/your/image.eps}
\end{slide-frame}



\end{document}
